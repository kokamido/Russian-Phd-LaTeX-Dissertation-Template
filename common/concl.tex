%% Согласно ГОСТ Р 7.0.11-2011:
%% 5.3.3 В заключении диссертации излагают итоги выполненного исследования, рекомендации, перспективы дальнейшей разработки темы.
%% 9.2.3 В заключении автореферата диссертации излагают итоги данного исследования, рекомендации и перспективы дальнейшей разработки темы.
\begin{enumerate}
  \item Разработаны методы численного анализа паттернов и переходных процессов в динамических системах с диффузией. 
  \item С помощью разработанных методов исследованы системы гликолитического осциллятора Хиггинса, гликолитического осциллятора Селькова и гликолитического осциллятора Селькова-Строгаца. Рассмотрены как детерминированные переходные процессы, так и поведение систем в присутствии случайного шума.
  \item Для рассмотренных моделей выявлены феномены и особенности генерации паттернов и переходных процессов, результаты опубликованы в рецензируемых изданиях, индексируемых в Scopus и Web of Science.
  \item Разработан программный комплекс, позволяющий проводить и анализировать численные эксперименты с описанными системами. Комплекс позволяет эффективно утилизировать вычислительные ресурсы и анализировать данные, полученные в большом количестве численных экспериментов. Разработанные программы зарегистрированы в реестре программ для ЭВМ.
\end{enumerate}

Представляет интерес дальнейшее развитие тематики классификации двумерных паттернов, вопрос о которой поднят в главе 3 основного содержания работы. В настоящее время существует разрыв между рассматриваемыми в литературе типами паттернов и подходами к их автоматической классификации, что затрудняет численный анализ генерируемых паттернов и требует большого объёма времени исследователя для выявления закономерностей в динамике образования паттернов в зависимости от параметров системы. Обобщение предложенного механизма на большее количество типов паттернов могло бы существенно сократить временные затраты на подобный анализ.